%% Generated by Sphinx.
\def\sphinxdocclass{report}
\documentclass[letterpaper,10pt,english]{sphinxmanual}
\ifdefined\pdfpxdimen
   \let\sphinxpxdimen\pdfpxdimen\else\newdimen\sphinxpxdimen
\fi \sphinxpxdimen=.75bp\relax

\PassOptionsToPackage{warn}{textcomp}
\usepackage[utf8]{inputenc}
\ifdefined\DeclareUnicodeCharacter
% support both utf8 and utf8x syntaxes
\edef\sphinxdqmaybe{\ifdefined\DeclareUnicodeCharacterAsOptional\string"\fi}
  \DeclareUnicodeCharacter{\sphinxdqmaybe00A0}{\nobreakspace}
  \DeclareUnicodeCharacter{\sphinxdqmaybe2500}{\sphinxunichar{2500}}
  \DeclareUnicodeCharacter{\sphinxdqmaybe2502}{\sphinxunichar{2502}}
  \DeclareUnicodeCharacter{\sphinxdqmaybe2514}{\sphinxunichar{2514}}
  \DeclareUnicodeCharacter{\sphinxdqmaybe251C}{\sphinxunichar{251C}}
  \DeclareUnicodeCharacter{\sphinxdqmaybe2572}{\textbackslash}
\fi
\usepackage{cmap}
\usepackage[T1]{fontenc}
\usepackage{amsmath,amssymb,amstext}
\usepackage{babel}
\usepackage{times}
\usepackage[Bjarne]{fncychap}
\usepackage{sphinx}

\fvset{fontsize=\small}
\usepackage{geometry}

% Include hyperref last.
\usepackage{hyperref}
% Fix anchor placement for figures with captions.
\usepackage{hypcap}% it must be loaded after hyperref.
% Set up styles of URL: it should be placed after hyperref.
\urlstyle{same}
\addto\captionsenglish{\renewcommand{\contentsname}{Contents:}}

\addto\captionsenglish{\renewcommand{\figurename}{Fig.\@ }}
\makeatletter
\def\fnum@figure{\figurename\thefigure{}}
\makeatother
\addto\captionsenglish{\renewcommand{\tablename}{Table }}
\makeatletter
\def\fnum@table{\tablename\thetable{}}
\makeatother
\addto\captionsenglish{\renewcommand{\literalblockname}{Listing}}

\addto\captionsenglish{\renewcommand{\literalblockcontinuedname}{continued from previous page}}
\addto\captionsenglish{\renewcommand{\literalblockcontinuesname}{continues on next page}}
\addto\captionsenglish{\renewcommand{\sphinxnonalphabeticalgroupname}{Non-alphabetical}}
\addto\captionsenglish{\renewcommand{\sphinxsymbolsname}{Symbols}}
\addto\captionsenglish{\renewcommand{\sphinxnumbersname}{Numbers}}

\addto\extrasenglish{\def\pageautorefname{page}}

\setcounter{tocdepth}{1}



\title{plataforma de videos Documentation}
\date{Oct 21, 2021}
\release{v1.0}
\author{Jonh Kennedy}
\newcommand{\sphinxlogo}{\vbox{}}
\renewcommand{\releasename}{Release}
\makeindex
\begin{document}

\pagestyle{empty}
\sphinxmaketitle
\pagestyle{plain}
\sphinxtableofcontents
\pagestyle{normal}
\phantomsection\label{\detokenize{index::doc}}

\index{usuario (module)@\spxentry{usuario}\spxextra{module}}\index{Usuario (class in usuario)@\spxentry{Usuario}\spxextra{class in usuario}}

\begin{fulllineitems}
\phantomsection\label{\detokenize{index:usuario.Usuario}}\pysiglinewithargsret{\sphinxbfcode{\sphinxupquote{class }}\sphinxcode{\sphinxupquote{usuario.}}\sphinxbfcode{\sphinxupquote{Usuario}}}{\emph{**kwargs}}{}
Modelo que representa usuario generico.
\begin{description}
\item[{Atributos}] \leavevmode
id: Integer

nome: Unicode

email: Unicode

senha: Unicode

tipo: Unicode

\end{description}

\end{fulllineitems}

\index{get\_usuario() (in module usuario)@\spxentry{get\_usuario()}\spxextra{in module usuario}}

\begin{fulllineitems}
\phantomsection\label{\detokenize{index:usuario.get_usuario}}\pysiglinewithargsret{\sphinxcode{\sphinxupquote{usuario.}}\sphinxbfcode{\sphinxupquote{get\_usuario}}}{}{}
Retorna o registro de um usuario
generico.
\begin{quote}\begin{description}
\item[{Returns}] \leavevmode
um arquivo json com o registro de um usuario generico.

\item[{Return type}] \leavevmode
json

\end{description}\end{quote}

\end{fulllineitems}

\phantomsection\label{\detokenize{index:module-aprendiz}}\index{aprendiz (module)@\spxentry{aprendiz}\spxextra{module}}\index{Aprendiz (class in aprendiz)@\spxentry{Aprendiz}\spxextra{class in aprendiz}}

\begin{fulllineitems}
\phantomsection\label{\detokenize{index:aprendiz.Aprendiz}}\pysiglinewithargsret{\sphinxbfcode{\sphinxupquote{class }}\sphinxcode{\sphinxupquote{aprendiz.}}\sphinxbfcode{\sphinxupquote{Aprendiz}}}{\emph{nome}, \emph{email}, \emph{senha}, \emph{data\_nasc}}{}
Modelo que representa usuario aprendiz.
\begin{description}
\item[{Atributos}] \leavevmode
id: Integer

nome: Unicode

email: Unicode

senha: Unicode

data\_nasc: Date

\end{description}

\end{fulllineitems}

\index{add\_Aprendiz() (in module aprendiz)@\spxentry{add\_Aprendiz()}\spxextra{in module aprendiz}}

\begin{fulllineitems}
\phantomsection\label{\detokenize{index:aprendiz.add_Aprendiz}}\pysiglinewithargsret{\sphinxcode{\sphinxupquote{aprendiz.}}\sphinxbfcode{\sphinxupquote{add\_Aprendiz}}}{}{}
Adiciona um usuario aprendiz na tabela
aprendiz.
\begin{quote}\begin{description}
\item[{Returns}] \leavevmode
Retorna um json com status da adicao

\item[{Return type}] \leavevmode
json

\end{description}\end{quote}

\end{fulllineitems}

\index{get\_Aprendiz() (in module aprendiz)@\spxentry{get\_Aprendiz()}\spxextra{in module aprendiz}}

\begin{fulllineitems}
\phantomsection\label{\detokenize{index:aprendiz.get_Aprendiz}}\pysiglinewithargsret{\sphinxcode{\sphinxupquote{aprendiz.}}\sphinxbfcode{\sphinxupquote{get\_Aprendiz}}}{}{}
Retorna o registro de um usuario
aprendiz.
\begin{quote}\begin{description}
\item[{Returns}] \leavevmode
Retorna um arquivo json com o registro de um usuario aprendiz.

\item[{Return type}] \leavevmode
json

\end{description}\end{quote}

\end{fulllineitems}

\index{list\_Aprendiz() (in module aprendiz)@\spxentry{list\_Aprendiz()}\spxextra{in module aprendiz}}

\begin{fulllineitems}
\phantomsection\label{\detokenize{index:aprendiz.list_Aprendiz}}\pysiglinewithargsret{\sphinxcode{\sphinxupquote{aprendiz.}}\sphinxbfcode{\sphinxupquote{list\_Aprendiz}}}{}{}
Lista todos os registros da tabela
aprendiz.
\begin{quote}\begin{description}
\item[{Returns}] \leavevmode
Retorna um arquivo json com todos os registros da tabela aprendiz.

\item[{Return type}] \leavevmode
json

\end{description}\end{quote}

\end{fulllineitems}

\phantomsection\label{\detokenize{index:module-business}}\index{business (module)@\spxentry{business}\spxextra{module}}\phantomsection\label{\detokenize{index:module-canal}}\index{canal (module)@\spxentry{canal}\spxextra{module}}\index{Canal (class in canal)@\spxentry{Canal}\spxextra{class in canal}}

\begin{fulllineitems}
\phantomsection\label{\detokenize{index:canal.Canal}}\pysiglinewithargsret{\sphinxbfcode{\sphinxupquote{class }}\sphinxcode{\sphinxupquote{canal.}}\sphinxbfcode{\sphinxupquote{Canal}}}{\emph{nome}, \emph{dono}, \emph{thumbnail}}{}
Modelo que representa um canal.
\begin{description}
\item[{Atributos}] \leavevmode
id: Integer

nome: Unicode

dono: Integer

thumbnail: LargeBinary

\end{description}

\end{fulllineitems}

\index{add\_canal() (in module canal)@\spxentry{add\_canal()}\spxextra{in module canal}}

\begin{fulllineitems}
\phantomsection\label{\detokenize{index:canal.add_canal}}\pysiglinewithargsret{\sphinxcode{\sphinxupquote{canal.}}\sphinxbfcode{\sphinxupquote{add\_canal}}}{}{}
Adiciona um canal na tabela
canal.
\begin{quote}\begin{description}
\item[{Returns}] \leavevmode
Retorna uma dupla com vazio e o codigo 200.

\item[{Return type}] \leavevmode
(str,int)

\end{description}\end{quote}

\end{fulllineitems}

\index{get\_canais\_aprendiz() (in module canal)@\spxentry{get\_canais\_aprendiz()}\spxextra{in module canal}}

\begin{fulllineitems}
\phantomsection\label{\detokenize{index:canal.get_canais_aprendiz}}\pysiglinewithargsret{\sphinxcode{\sphinxupquote{canal.}}\sphinxbfcode{\sphinxupquote{get\_canais\_aprendiz}}}{}{}
Lista todos os registros
de canais nos quais um usuario
aprendiz esta inscrito.
\begin{quote}\begin{description}
\item[{Returns}] \leavevmode
Retorna um arquivo json com registros de canais.

\item[{Return type}] \leavevmode
json

\end{description}\end{quote}

\end{fulllineitems}

\index{get\_canais\_business() (in module canal)@\spxentry{get\_canais\_business()}\spxextra{in module canal}}

\begin{fulllineitems}
\phantomsection\label{\detokenize{index:canal.get_canais_business}}\pysiglinewithargsret{\sphinxcode{\sphinxupquote{canal.}}\sphinxbfcode{\sphinxupquote{get\_canais\_business}}}{}{}
Lista todos os registros
de canais pertencentes a um usuario
business.
\begin{quote}\begin{description}
\item[{Returns}] \leavevmode
Retorna um arquivo json com registros de canais.

\item[{Return type}] \leavevmode
json

\end{description}\end{quote}

\end{fulllineitems}

\index{rm\_canal() (in module canal)@\spxentry{rm\_canal()}\spxextra{in module canal}}

\begin{fulllineitems}
\phantomsection\label{\detokenize{index:canal.rm_canal}}\pysiglinewithargsret{\sphinxcode{\sphinxupquote{canal.}}\sphinxbfcode{\sphinxupquote{rm\_canal}}}{}{}
Remove um registro de canal da tabela
canal.

\end{fulllineitems}

\phantomsection\label{\detokenize{index:module-inscricao}}\index{inscricao (module)@\spxentry{inscricao}\spxextra{module}}\index{Inscricao (class in inscricao)@\spxentry{Inscricao}\spxextra{class in inscricao}}

\begin{fulllineitems}
\phantomsection\label{\detokenize{index:inscricao.Inscricao}}\pysiglinewithargsret{\sphinxbfcode{\sphinxupquote{class }}\sphinxcode{\sphinxupquote{inscricao.}}\sphinxbfcode{\sphinxupquote{Inscricao}}}{\emph{canal}, \emph{aprendiz}}{}
Modelo que representa uma inscricao.
\begin{description}
\item[{Atributos}] \leavevmode
id: Integer

canal: Unicode

aprendiz: Unicode

\end{description}

\end{fulllineitems}

\phantomsection\label{\detokenize{index:module-assinatura}}\index{assinatura (module)@\spxentry{assinatura}\spxextra{module}}\index{Assinatura (class in assinatura)@\spxentry{Assinatura}\spxextra{class in assinatura}}

\begin{fulllineitems}
\phantomsection\label{\detokenize{index:assinatura.Assinatura}}\pysiglinewithargsret{\sphinxbfcode{\sphinxupquote{class }}\sphinxcode{\sphinxupquote{assinatura.}}\sphinxbfcode{\sphinxupquote{Assinatura}}}{\emph{usuario}, \emph{plano}}{}
Modelo que representa assinatura.
\begin{description}
\item[{Atributos}] \leavevmode
id: Integer

usuario: Unicode

plano: Unicode

\end{description}

\end{fulllineitems}

\phantomsection\label{\detokenize{index:module-plano}}\index{plano (module)@\spxentry{plano}\spxextra{module}}\index{Plano (class in plano)@\spxentry{Plano}\spxextra{class in plano}}

\begin{fulllineitems}
\phantomsection\label{\detokenize{index:plano.Plano}}\pysiglinewithargsret{\sphinxbfcode{\sphinxupquote{class }}\sphinxcode{\sphinxupquote{plano.}}\sphinxbfcode{\sphinxupquote{Plano}}}{\emph{**kwargs}}{}
Modelo que representa um plano.
\begin{description}
\item[{Atributos}] \leavevmode
id: Integer

nome: Unicode

preco: Float

\end{description}

\end{fulllineitems}

\phantomsection\label{\detokenize{index:module-video}}\index{video (module)@\spxentry{video}\spxextra{module}}\index{Video (class in video)@\spxentry{Video}\spxextra{class in video}}

\begin{fulllineitems}
\phantomsection\label{\detokenize{index:video.Video}}\pysiglinewithargsret{\sphinxbfcode{\sphinxupquote{class }}\sphinxcode{\sphinxupquote{video.}}\sphinxbfcode{\sphinxupquote{Video}}}{\emph{canal}, \emph{titulo}, \emph{descricao}, \emph{link}}{}
Modelo que representa video.
\begin{description}
\item[{Atributos}] \leavevmode
id: Integer

canal: Integer

titulo: Unicode

descricao: Unicode

link: Unicode

\end{description}

\end{fulllineitems}

\index{add\_video() (in module video)@\spxentry{add\_video()}\spxextra{in module video}}

\begin{fulllineitems}
\phantomsection\label{\detokenize{index:video.add_video}}\pysiglinewithargsret{\sphinxcode{\sphinxupquote{video.}}\sphinxbfcode{\sphinxupquote{add\_video}}}{}{}
Adiciona o registro de um video na tabela
video.
\begin{quote}\begin{description}
\item[{Returns}] \leavevmode
retorna uma dupla com vazio e o codigo 200.

\item[{Return type}] \leavevmode
(str,int)

\end{description}\end{quote}

\end{fulllineitems}

\index{edit\_video() (in module video)@\spxentry{edit\_video()}\spxextra{in module video}}

\begin{fulllineitems}
\phantomsection\label{\detokenize{index:video.edit_video}}\pysiglinewithargsret{\sphinxcode{\sphinxupquote{video.}}\sphinxbfcode{\sphinxupquote{edit\_video}}}{}{}
Edita atributos de um registro video
da tabela videos.
\begin{quote}\begin{description}
\item[{Returns}] \leavevmode
uma dupla com vazio e o codigo 200.

\item[{Return type}] \leavevmode
(str,int)

\end{description}\end{quote}

\end{fulllineitems}

\index{rm\_video() (in module video)@\spxentry{rm\_video()}\spxextra{in module video}}

\begin{fulllineitems}
\phantomsection\label{\detokenize{index:video.rm_video}}\pysiglinewithargsret{\sphinxcode{\sphinxupquote{video.}}\sphinxbfcode{\sphinxupquote{rm\_video}}}{}{}
Remove um registro de um video da tabela
video.
\begin{quote}\begin{description}
\item[{Returns}] \leavevmode
um json vazio

\item[{Return type}] \leavevmode
json

\end{description}\end{quote}

\end{fulllineitems}

\index{Business (class in business)@\spxentry{Business}\spxextra{class in business}}

\begin{fulllineitems}
\phantomsection\label{\detokenize{index:business.Business}}\pysiglinewithargsret{\sphinxbfcode{\sphinxupquote{class }}\sphinxcode{\sphinxupquote{business.}}\sphinxbfcode{\sphinxupquote{Business}}}{\emph{nome}, \emph{email}, \emph{senha}}{}
Modelo que representa usuario business.
\begin{description}
\item[{Atributos}] \leavevmode
id: Integer

nome: Unicode

email: Unicode

senha: Unicode

tipo: Unicode

\end{description}

\end{fulllineitems}

\index{add\_Business() (in module business)@\spxentry{add\_Business()}\spxextra{in module business}}

\begin{fulllineitems}
\phantomsection\label{\detokenize{index:business.add_Business}}\pysiglinewithargsret{\sphinxcode{\sphinxupquote{business.}}\sphinxbfcode{\sphinxupquote{add\_Business}}}{}{}
Adiciona um usuario business na tabela
business.
\begin{quote}\begin{description}
\item[{Returns}] \leavevmode
uma dupla com vazio e o codigo 200.

\item[{Return type}] \leavevmode
(str,int)

\end{description}\end{quote}

\end{fulllineitems}

\index{del\_Business() (in module business)@\spxentry{del\_Business()}\spxextra{in module business}}

\begin{fulllineitems}
\phantomsection\label{\detokenize{index:business.del_Business}}\pysiglinewithargsret{\sphinxcode{\sphinxupquote{business.}}\sphinxbfcode{\sphinxupquote{del\_Business}}}{}{}
Remove um usuario business da tabela
business.

\end{fulllineitems}

\index{edit\_Business() (in module business)@\spxentry{edit\_Business()}\spxextra{in module business}}

\begin{fulllineitems}
\phantomsection\label{\detokenize{index:business.edit_Business}}\pysiglinewithargsret{\sphinxcode{\sphinxupquote{business.}}\sphinxbfcode{\sphinxupquote{edit\_Business}}}{}{}
Edita atributos de um registro business
da tabela business.
\begin{quote}\begin{description}
\item[{Returns}] \leavevmode
uma dupla com vazio e o codigo 200.

\item[{Return type}] \leavevmode
(str,int)

\end{description}\end{quote}

\end{fulllineitems}

\index{get\_bussiness() (in module business)@\spxentry{get\_bussiness()}\spxextra{in module business}}

\begin{fulllineitems}
\phantomsection\label{\detokenize{index:business.get_bussiness}}\pysiglinewithargsret{\sphinxcode{\sphinxupquote{business.}}\sphinxbfcode{\sphinxupquote{get\_bussiness}}}{}{}
Retorna o registro de um usuario
business.
\begin{quote}\begin{description}
\item[{Returns}] \leavevmode
um arquivo json com o registro de um usuario business.

\item[{Return type}] \leavevmode
json

\end{description}\end{quote}

\end{fulllineitems}

\index{list\_Business() (in module business)@\spxentry{list\_Business()}\spxextra{in module business}}

\begin{fulllineitems}
\phantomsection\label{\detokenize{index:business.list_Business}}\pysiglinewithargsret{\sphinxcode{\sphinxupquote{business.}}\sphinxbfcode{\sphinxupquote{list\_Business}}}{}{}
Lista todos os registros da tabela
business.
\begin{quote}\begin{description}
\item[{Returns}] \leavevmode
um arquivo json com todos os registros da tabela business.

\item[{Return type}] \leavevmode
json

\end{description}\end{quote}

\end{fulllineitems}



\chapter{Indices and tables}
\label{\detokenize{index:indices-and-tables}}\begin{itemize}
\item {} 
\DUrole{xref,std,std-ref}{genindex}

\item {} 
\DUrole{xref,std,std-ref}{modindex}

\item {} 
\DUrole{xref,std,std-ref}{search}

\end{itemize}


\renewcommand{\indexname}{Python Module Index}
\begin{sphinxtheindex}
\let\bigletter\sphinxstyleindexlettergroup
\bigletter{a}
\item\relax\sphinxstyleindexentry{aprendiz}\sphinxstyleindexpageref{index:\detokenize{module-aprendiz}}
\item\relax\sphinxstyleindexentry{assinatura}\sphinxstyleindexpageref{index:\detokenize{module-assinatura}}
\indexspace
\bigletter{b}
\item\relax\sphinxstyleindexentry{business}\sphinxstyleindexpageref{index:\detokenize{module-business}}
\indexspace
\bigletter{c}
\item\relax\sphinxstyleindexentry{canal}\sphinxstyleindexpageref{index:\detokenize{module-canal}}
\indexspace
\bigletter{i}
\item\relax\sphinxstyleindexentry{inscricao}\sphinxstyleindexpageref{index:\detokenize{module-inscricao}}
\indexspace
\bigletter{p}
\item\relax\sphinxstyleindexentry{plano}\sphinxstyleindexpageref{index:\detokenize{module-plano}}
\indexspace
\bigletter{u}
\item\relax\sphinxstyleindexentry{usuario}\sphinxstyleindexpageref{index:\detokenize{module-usuario}}
\indexspace
\bigletter{v}
\item\relax\sphinxstyleindexentry{video}\sphinxstyleindexpageref{index:\detokenize{module-video}}
\end{sphinxtheindex}

\renewcommand{\indexname}{Index}
\printindex
\end{document}